\CWHeader{Лабораторная работа №8}

\CWProblem{
Бычкам дают пищевые добавки, чтобы ускорить их рост. Каждая добавка содержит некоторые из N действующих веществ. Соотношения количеств веществ в добавках могут отличаться.
Воздействие добавки определяется как $c_1a_1 + c_2a_2 +$...$+c_N_a_N$, где $a_i$ — количество i-го вещества в добавке, $c_i$ – неизвестный коэффициент, связанный с веществом и не зависящий от добавки. Чтобы найти неизвестные коэффициенты $c_i$, Биолог может измерить воздействие любой добавки, использовав один её мешок. Известна цена мешка каждой из M ($M \leq N$) различных добавок. Нужно помочь Биологу подобрать самый дешевый наобор добавок, позволяющий найти коэффициенты $c_i$. Возможно, соотношения веществ в добавках таковы, что определить коэффициенты нельзя.
}
\pagebreak
