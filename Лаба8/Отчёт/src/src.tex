\section{Описание}
Требуется реализовать жадный алгоритм, решающий поставленную задачу: найти набор добавок минимальной стоимости. Для решения этой задачи используются методы из линейной алгебры. Нам требуется найти минимальную систему линейно независимых уравнений, которая имела бы одно решение. Для этого нам требуется среди всех уравнений найти те, которые бы были друг другу линейно независимы. Реализуется это также, как и в линейной алгебре: путём вычетов строчек матрицы. Главная идея заключается в том, что в начале мы сортируем все уравнения в порядке возрастания цены и каждый раз берём уравнения с наименьшей ценой, которое ещё не было использовано. Если оно будет линейно независимо для уже составленной системы, то мы его берём и продолжаем поиск нового уравнения. Делать мы это можем потому, что заменить линейно независимое уравнение в системе мы можем только линейно зависимым ему же, значит можно спокойно брать самые дешёвые уравнения.

\pagebreak

\section{Исходный код}
Как уже было сказано выше, изначально мы сортируем все добавки по возрастанию цены. В векторе $discharge$ на позиции i находится номер уравнения из вектора $preAns$, который отвечает за "ступеньку"\;матрицы, которая находится в i-том столбце. Получая новую строку, ищем в ней первый ненулевой элемент. Если на этой позиции в матрице ещё нет "ступеньки"\;, то добавляем уравнение в вектор $preAns$ и добавляем его номер в $discharge$. Иначе, с помощью вычетов ищем место для текущего уравнения, на котором ещё нет "ступеньки". Если такое место найдено не было, то уравнение для текущей системы является линейно зависимым и оно нам не подходит. Подбор уравнений продолжается до тех пор, пока количество уравнений не равно количеству неизвестных или пока все уравнения не просмотрены. 

\begin{lstlisting}[language=C]
#include <iostream>
#include <vector>
#include <algorithm>

using namespace std;

double eps = 1;

void CalculateMachineEps() {
    while (1 + eps > 1) {
        eps = eps / 2;
    }
}

vector<double> Difference(vector<double> v, vector<double> v1, int tmp) { // приведение к ортогональному виду
    double coef = v[tmp] / v1[tmp];
    for (int i = tmp; i < v.size();i++) {
        v[i] = v[i] - v1[i] * coef;
    }
    return v;
}

int main() {
    ios::sync_with_stdio(false);
    cin.tie(nullptr); cout.tie(nullptr);
    int n, m;
    CalculateMachineEps();
    cin >> n >> m;
    vector<vector<double>> input(n, vector<double>(m));
    vector<pair<int, int>> price(n);
    for (int i = 0;i < n;i++) {
        for (int j = 0;j < m;j++) {
            cin >> input[i][j];
        }
        cin >> price[i].first;
        price[i].second = i;
    }
    sort(price.begin(), price.end());
    vector<int> discharge(m, -1);  // ступени ортогональной матрицы
    vector<double> v(m);
    int tmp;
    vector<vector<double>> preAns;
    for (int i = 0;i < n;i++) {  // берём самые дешёвые строки
        v = input[price[i].second];
        tmp = -1;
        for (int j = 0;j < m;j++) {
            if (v[j] > eps) {
                tmp = j;
                break;
            }
        }
        if (tmp == -1) {  // вся строка нулевая
            price[i] = { 1e3, 1e3 };
        }
        else if (discharge[tmp] == -1) {
            discharge[tmp] = preAns.size();
            preAns.push_back(v);
        }
        else {
            while (tmp != -1 && discharge[tmp] != -1) {
                v = Difference(v, preAns[discharge[tmp]], tmp);
                tmp = -1;
                for (int j = 0;j < m;j++) {
                    if (v[j] > eps || v[j] < -eps) {
                        tmp = j;
                        break;
                    }
                }
            }
            if (tmp == -1) {  // строка линейно зависима от тех, которые уже находятся в матрице
                price[i] = { 1e3, 1e3 };
            }
            else {  // подходящая строка
                discharge[tmp] = preAns.size();
                preAns.push_back(v);
            }
        }
        if (preAns.size() == m) { // квадратная матрица заполнена
            break;
        }
    }
    if (preAns.size() != m) {
        cout << -1;
    }
    else {
        vector<int> ans;
        sort(price.begin(), price.end());
        for (int i = 0;i < m;i++) {  // выписываем номера
            ans.push_back(price[i].second + 1);
        }
        sort(ans.begin(), ans.end());
        for (int i = 0;i < m;i++) {
            cout << ans[i] << ' ';
        }
    }
}
\end{lstlisting}


\section{Консоль}
\begin{alltt}
root@DESKTOP-5HM2HTK:~# cat <test
3 3
1 0 2 3
1 0 2 4
2 0 1 2
root@DESKTOP-5HM2HTK:~# g++ lab8.cpp
root@DESKTOP-5HM2HTK:~# ./a.out <test
-1
root@DESKTOP-5HM2HTK:~#
\end{alltt}
\pagebreak

