\section{Тест производительности}

\begin{alltt}
root@DESKTOP-5HM2HTK:~# g++ generator5.cpp
root@DESKTOP-5HM2HTK:~# ./a.out tests
root@DESKTOP-5HM2HTK:~# g++ benchmark.cpp
root@DESKTOP-5HM2HTK:~# ./a.out <tests/1.t
Size of field is: 1000 X 1000
Number of requests is: 5
A star time: 1370ms
BFS time: 3458ms
root@DESKTOP-5HM2HTK:~#
\end{alltt}

Как видно, алгоритм $A$$^*$ работает почти в три раза быстрее, чем стандратный обход в ширину на поле размером $1000 \times 1000$ . И это всего лишь на 5-ти тестах. Такое сильное ускорение получается из-за того, что в алгоритме $A$$^*$ используется функция $f(u)$, которая позволяет сразу найти направление, в которую следует двигаться. Если на пути не встретиться значительных преград, то данный алгоритм даёт огромное преимущество перед обходом в ширину.

\pagebreak

