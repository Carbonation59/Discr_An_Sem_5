\CWHeader{Курсовой проект: усложнённый вариант}

\CWProblem{
Реализуйте систему для поиска пути в графе дорог с использованием эвристических алгоритмов.\\*
\textbf{Формат входных данных:}\\*
./prog preprocess –nodes <nodes file> –edges <edges file> –output <preprocessed
graph>\\*
Входной файл с перекрёстками, входной файл с дорогами, выходной файл с графом.
./prog search –graph <preprocessed graph> –input <input file> -output <output file>
[–full-output]\\*
Входной файл с графом, входной файл с запросами, выходной файл с ответами на
запросы, переключение формата выходного файла на подробный.\\*
\textbf{Формат результата:}\\*
Если опция –full-output не указана: на каждый запрос в отдельной строке выводится длина кратчайшего пути между заданными вершинами с относительной погрешностью не более 1e-6.\\*
Если опция –full-output указана: на каждый запрос выводится отдельная строка, с
длиной кратчайшего пути между заданными вершинами с относительной погрешностью не более 1e-6, а затем сам путь в формате как в файле рёбер. Расстояние между точками следует вычислять как расстояние между точками на сфере с радиусом 6371км, если пути между точками нет,
вывести -1 и длину пути в вершинах 0.
}
\pagebreak
