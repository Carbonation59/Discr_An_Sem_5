\section{Тест производительности}

\begin{alltt}
root@DESKTOP-5HM2HTK:~# g++ generator.cpp
root@DESKTOP-5HM2HTK:~# ./a.out tests
root@DESKTOP-5HM2HTK:~# g++ benchmark.cpp
root@DESKTOP-5HM2HTK:~# ./a.out <tests/1.t
Number of nodes is: 1000
Number of edges is: 10000
A star time: 41ms
Dijkstra's algorithm time: 397ms
root@DESKTOP-5HM2HTK:~#
\end{alltt}

Как видно, алгоритм $A$$^*$ работает почти в десять раза быстрее, чем алгоритм Дейкстры на указанных данных. Такое сильное ускорение получается из-за того, что в алгоритме $A$$^*$ используется функция $f(u)$, которая позволяет сразу найти направление, в которую следует двигаться. Если на пути не встретиться значительных преград, то данный алгоритм даёт огромное преимущество перед другими алгоритмами.

\pagebreak

