\section{Тест производительности}

\begin{alltt}
root@DESKTOP-5HM2HTK:~# g++ -pedantic -Wall benchmark.cpp
root@DESKTOP-5HM2HTK:~# ./a.out <tests/1.t
Count of lines is 84
Suffix Tree time: 2477ms
Square algorithm time: 7868ms
root@DESKTOP-5HM2HTK:~#
\end{alltt}

Как видно, суффиксное дерево на тестах с длинной строки $ \approx $\ 10000 работает больше чем в 3 раза быстрее, чем алгоритм за квадрат. Из-за того, что алгоритм суффиксного дерева требует сначала построение самого дерева, и лишь потом осуществляет поиск, в его сложности присутствует некоторая константа, которая, при увеличении длины строки, будет всё меньше ощущаться.

\pagebreak

