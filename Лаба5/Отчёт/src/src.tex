\section{Описание}
Требуется написать реализацию суффиксного дерева. Оно представляет из себя сжатый $Trie$, который содержит в себе все суффиксы строки, которая была "скормлена" \ дереву. Алгоритм Укконена за счёт своих ускорений (суффиксные ссылки, прыжки по счётчику, хранение на рёбрах начала и конца отрезка вместо всего куска строки, добавление строки по префиксам) даёт возможность построить суффиксное дерево за $O(n)$, где $n$ - длина строки. Суффиксное дерево позволяет решать большое количество различных задач. Например, если в лабораторной работе №4 нам нужно было найти указанный паттерн в разных текстах, то с помощью суффиксного дерева мы можем загнать какой-то текст, и уже в нём искать любые паттерны за $O(m)$, где m - длина паттерна. В моей лабораторной работе использован для решения поставленной задачи принцип циклического буфера. К изначальной строке в конец была "приклеена" \ она же, и уже по этой строке мы строим суффиксное дерево. Такой ход даём нам возможность найти в суффиксном дереве изначальную строку, прочитанную циклически с любой позиции. Остаётся лишь найти минимальную лексикографискую строку, которую мы можем получить таким образом, проходя по наименьшим буквам из всех возможных в каждом узле.

\pagebreak

\section{Исходный код}
Терминальный символ, который ставится в конце строки перед её добавлением в суффиксное дерево, здесь берётся как самый большой лексикографический символ (т.к. в ответ он входить не должен). Каждая нода состоит из мапы, 
которая отражает переход в другие узлы по всем возможным буквам, суффиксную ссылку из данного узла и пару, которая отражает отрезок строки, покрываемый ребром, ведущим в данную ноду. При добавлении нового префикса, учитывается уже пройденный путь при добавлении прошлых префиксов. Если продолжить путь по дереву мы не можем, а пройденный путь у нас итак нулевой, то просто добавим новый лист. Если же пройденный путь не нулевой, то добавим текущую букву требуемым способом (либо на ребро, если мы стоим где-то на ребре, либо создадим новый лист, если мы прошли всё ребро). Далее идёт переход по суффиксным ссылкам, или проход этого же пути, только "обрезанного" \ спереди на один символ, и повторение тех же действий. Помимо этого, при проходе нужно создавать суффиксные ссылки для вершин, у которых их нет. При выводе ответа идём по дереву, постоянно проходя по самой наименьшей лексокографически букве, по которой мы можем перейти. Путь наш должен составлять ровно длину данной нам строки, поэтому он может закончиться даже где-то на ребре, а не только на его конце.

\begin{longtable}{|p{7.5cm}|p{7.5cm}|}
\hline
\rowcolor{lightgray}
\multicolumn{2}{|c|} {lab2.cpp}\\
\hline
TSuffixTree::~TSuffixTree()& Деструктор дерева\\
\hline
void TSuffixTree::Delete(TSuffixNode* node)& Рекурсивное удаление дерева\\
\hline
void TSuffixTree::Build() & Построение дерева\\
\hline
void TSuffixTree::AddSuffixes()& Добавление каждого нового префикса строки в дерево\\
\hline
void TSuffixTree::Print(TSuffixNode* node, int k)& Печать ответа\\
\hline
\rowcolor{lightgray}
\hline
\end{longtable}

\begin{lstlisting}[language=C]
const char TERMINAL = 'z' + 1;
const int ABSOLUTE_END = 1e7;
int endOfPrefix = -1; // конец префикса в  индексации
int remainingPart = 0; // пройденный путь
int remainingPass = 0; // пройденный путь на ребре
string pattern; 

class TSuffixNode {
public:
    map<char, TSuffixNode*> mapData;
    TSuffixNode* SuffixLink = nullptr;
    pair<int,int> Key = {-1,-1};
};

TSuffixNode* node = nullptr;  // текущая вершина
pair<int, int> keyNode;  // текущее ребро
char alphaNode;  // текущий ключ ребра
TSuffixNode* suffLink = nullptr; // текущая суффиксная ссылка
int suffNumber = -1;  // позиция буквы на которую нас перемещает суффиксная ссылка

class TSuffixTree {
public:
    TSuffixTree(const string pattern) { Build(); }
    ~TSuffixTree();
private:
    void Build();
    void AddSuffixes();
    void Print(TSuffixNode*, int);
    void Delete(TSuffixNode*);
private:
    TSuffixNode* Root = nullptr;
};
\end{lstlisting}


\section{Консоль}
\begin{alltt}
root@DESKTOP-5HM2HTK:~# cat <test
xabcd
root@DESKTOP-5HM2HTK:~# g++ lab5.cpp
root@DESKTOP-5HM2HTK:~# ./a.out <test
abcdx
root@DESKTOP-5HM2HTK:~#
\end{alltt}
\pagebreak

